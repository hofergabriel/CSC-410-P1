\documentclass[12pt]{article}
\usepackage{listings}
\usepackage{pdfpages}
%\usepackage[legalpaper, margin=1in]{geometry}
\begin{document}
\begin{titlepage}
   \begin{center}
       \vspace*{1cm}
       \Large
       Programming Assignment 1
       \normalsize

       \vspace{0.5cm}

       Author: Gabriel Hofer

       \vspace{0.5cm}

       CSC-410 Parallel Computing 

       \vspace{0.5cm}

       Instructor: Dr. Karlsson

       \vspace{0.5cm}

       Due: September 18, 2020

       \vfill

       Computer Science and Engineering\\
       South Dakota School of Mines and Technology\\
   \end{center}
\end{titlepage}
%------------------------------------------------------------------------------------


\newpage
\section*{Sieve of Eratosthenes}
\begin{lstlisting}[frame=single,language=C,caption=Non-parallelized Sieve of Eratosthenes (prime.c) \label{code:prime-sieve-nonparallelized}]
void erat(long int n, long int * pcnt){
  for(int i=2;i<=n;i++)
    sieve[i]=1;
  for(int i=2;i*i<=n;i++)
    if(sieve[i])
      for(int j=i*i;j<=n;j+=i)
        sieve[j]=0;
  *pcnt=0;
  for(int i=2;i<=n;i++)
    if(sieve[i])
      primes[(*pcnt)++]=i; 
}
\end{lstlisting}
\begin{lstlisting}[frame=single,language=C,caption=Parllelized Sieve of Eratosthenes (prime.c) \label{code:prime-sieve-parallelized}]
void erat2(long int n, long int * pcnt){
  #pragma omp parallel for
  for(int i=2;i<=n;i++)
    sieve[i]=1;
  int sqrtn = sqrt((double)n);
  for(int i=2;i <= sqrtn;i++)
    if(sieve[i]){
      #pragma omp parallel for
      for(int j=i*i;j<=n;j+=i)
        sieve[j]=0;
    }
  *pcnt=0;
  for(int i=2;i<=n;i++)
    if(sieve[i])
      primes[(*pcnt)++]=i; 
}
\end{lstlisting}
\newpage
\begin{lstlisting}[frame=single,language=C,caption=Measuring Runtime Performance (prime.c) \label{code:prime-sieve-runtime}]
void main(){
  long int n, pcnt;
  double start, end;
  scanf("%li",&n);
  n--;

  pcnt=0;
  start = omp_get_wtime();
  erat(n,& pcnt);
  end = omp_get_wtime();
  print(pcnt);
  printf("Elapsed time = %.6f seconds\n\n", end-start);

  // reset primes and sieve.
  for(int i=0; i<(1<<30); i++){
    sieve[i]=0;
    primes[i]=0;
  }

  pcnt=0;
  start = omp_get_wtime();
  erat2(n,& pcnt);
  end = omp_get_wtime();
  print(pcnt);
  printf("Elapsed time = %.6f seconds\n\n", end-start);
}
\end{lstlisting}
\begin{lstlisting}[frame=single,language=Bash,caption=Output in Terminal from prime sieve program (prime.c) \label{code:prime-sieve-terminal-output}]
\$ gcc prime.c -lm -fopenmp
\end{lstlisting}
%%%%%%%%%%%%%%%%%%%%%%%%%%%%%%%%%%%%%%%%%
\newpage
\section*{Monte Carlo Method}
\begin{lstlisting}[frame=single,language=Python,caption=Non-parallelized Monte Carlo Method (monte.c) \label{code:monte-carlo-nonparallelized}]
double monte(long long n){
  long long hits=0; 
  double x, y, pi;
  for(int i=0; i<n; i++)
    hits += sq((double)rand()/((double)RAND_MAX)) + 
      sq((double)rand()/((double)RAND_MAX)) 
      <= 1.0 ? 1 : 0;
  pi = 4.0*hits/(double)n;
  return pi;
}
\end{lstlisting}
\begin{lstlisting}[frame=single,language=Python,caption=Parallelized Monte Carlo Method (monte.c) \label{code:monte-carlo-parallelized}]
double monte2(long long n){
  long long hits=0; 
  double x, y, pi;
  #pragma omp parallel for
  for(int i=0; i<n; i++)
    hits += sq((double)rand()/((double)RAND_MAX)) + 
      sq((double)rand()/((double)RAND_MAX)) 
      <= 1.0 ? 1 : 0;
  pi = 4.0*hits/(double)n;
  return pi;
}
\end{lstlisting}
\newpage
\begin{lstlisting}[frame=single,language=Bash,caption=Measuring Runtime Performance (monte.c) \label{code:monte-carlo-runtime}]
void main(){
  long long n;
  double start, end, _PI_;
  scanf("%lld", & n);

  printf("Monte Carlo Method NON-parallelized\n");
  start = omp_get_wtime();
  _PI_=monte(n);
  end = omp_get_wtime();

  printf("PI: %f\n",_PI_);
  printf("Elapsed time = %f seconds\n\n", end-start);

  printf("Monte Carlo Method parallelized\n");
  start = omp_get_wtime();
  _PI_=monte2(n);
  end = omp_get_wtime();

  printf("PI: %f\n",_PI_);
  printf("Elapsed time = %f seconds\n\n", end-start);
}
\end{lstlisting}
\begin{lstlisting}[frame=single,language=Bash,caption=Output in Terminal (monte.c) \label{code:monte-carlo-terminal-output}]
\$ gcc monte.c -lm -fopenmp
\end{lstlisting}

\newpage
\subsection*{Description of Programs}
Functions erat and erat2 compute all of the primes between 1 and n. 
Functions monte and monte2 compute an approximation of PI.
erat and monte are non-parallelized while erat2 and monte2 are parallelized.

\subsection*{Description of Algorithms and Libraries: } 
I used OpenMP for both programs.  

The prime sieve program is essentially two nested for loops. Initially, we start with an 
array, called sieve, containing a True value in every cell. Then, we iterate i from 2 to
sqrt(n). For each of these iterations, we also iterate j from i*i to n in steps of i. At
each step, we set the current cell in sieve to False, because we know the number isn't prime.

The Monte Carlo Algorithm is a single for loop. Each iteration of the for loop is analogous 
to throwing a dart at a dart board. In our code, we use a circle of radius 1 as our dart board.
We call the C rand() function twice for the x and y dimensions. Pythagora's Theorem is used
to calculate whether the dart is more than 1.0 from the center of the board. If not, we
increment our count variable, hits. Otherwise, we continue to the next iteration of the loop
and throw another dart.

\subsection*{Description of Functions and Program Structure: } 
I created two files called prime.c and monte.c. 
The prime sieve is written in prime.c and, the Monte Carlo method is written in monte.c.  

\subsection*{How to compile and use the programs}
On Linux (Ubuntu): \\
\$ gcc filename.c -fopenmp -lm 

\subsection*{Description of the Testing and Verification Process: }
I viewed the output of both prime.c and monte.c to check the 
correctness and verified that the algorithms for both programs are correct.  
For testing the runtime of the programs and algorithms, I used the function omp\_get\_wtime() .

\subsection*{Description of Deliverable: }
I submitted a zip folder with the following: 
\begin{enumerate}
  \item this pdf document
  \item prime.c
  \item monte.c
\end{enumerate}
\end{document}

